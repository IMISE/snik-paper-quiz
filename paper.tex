% add. options: [seceqn,secthm,crcready,onecolumn]
\documentclass[sw]{iosart2x}

\usepackage{booktabs}
\usepackage{tabulary}
\usepackage{csquotes}
\usepackage{siunitx}
%\usepackage{graphicx}
\usepackage{todonotes}
%\usepackage{natbib} % does not seem to work with ios 1
\renewcommand{\citet}{\cite}% citet is not defined without natbib
\renewcommand{\citep}{\cite}% citep is not defined without natbib
\setlength{\marginparwidth}{2.1cm}% enough space for todonotes
\usepackage{aurl}
\daurl{bb}{http://www.snik.eu/ontology/bb/}
\usepackage{listings}
\lstset{language=SPARQL,breaklines=true}
%\usepackage{cleveref}
%%%%%%%%%%% End of definitions

\pubyear{2019}
\volume{0}
\firstpage{1}
\lastpage{1}

\begin{document}

\begin{frontmatter}

%\pretitle{}
\title{SNIK Quiz: A Game based on an Ontology of Hospital Information Management}
\runningtitle{SNIK Quiz: A Game based on an Ontology of Hospital Information Management}
%\subtitle{}

% Two or more authors:
\author[A]{\inits {K.}\fnms{Konrad} \snm{Höffner}\ead[label=e1]{konrad.hoeffner@imise.uni-leipzig.de}%
\thanks{Corresponding author. \printead{e1}.}},
%\todo{Wer möchte mitmachen? Ich habe die Einträge schonmal angelegt.}
\author[A]{\fnms{Franziska} \snm{Jahn}\ead[label=e2]{franziska.jahn@imise.uni-leipzig.de}},
\author[A]{\fnms{Birgit} \snm{Schneider}\ead[label=e3]{birgit.schneider@imise.uni-leipzig.de}},
\author[A]{\fnms{Anna} \snm{Lörke}\ead[label=e4]{anna.loerke@imise.uni-leipzig.de}},
\author[A]{\fnms{Thomas} \snm{Pause}\ead[label=e5]{thomas.pause@imise.uni-leipzig.de}},
\author[A]{\fnms{Alfred} \snm{Winter}\ead[label=e6]{alfred.winter@imise.uni-leipzig.de}}
\runningauthor{K. Höffner et al.}
\address[A]{Institute for Medical Informatics, Statistics and Epidemiology (IMISE),
\orgname{University of Leipzig}, \cny{Germany}\printead[presep={\\}]{e1,e2,e3,e4,e5,e6}}
%Medical Informatics, Management of Health Information Systems
%Härtelstraße 16--18, D-04107 Leipzig

\begin{abstract}
Hospital Information Management is a central topic for students of medical informatics.
SNIK is an ontology of Information Management in hospitals generated by manually extracting textbooks and other knowledge sources that is used primarily for teaching.
The DBpedia-powered Clover Quiz shows that ontologies can be used to automatically generate multiple-choice questions.
We apply this approach to SNIK and generate NUMBER English questions of NUMBER types.
Students have answered NUMBER questions and rated the game as NUMBER out of NUMBER.
%Students can explore SNIK in the graph-based visualization SNIK Graph.
SNIK Quiz is freely available as an open source web application.
\end{abstract}


\begin{keyword}
\kwd{information management, information systems, hospital information management,multiple choice question,e-learning}
\end{keyword}

\end{frontmatter}

\section{Introduction}\label{sec:introduction}
\todo{Describe the meta model here}\todo{this is from the swj paper, rewrite and adapt}
The \textbf{Semantic Network of Information Management in Hospitals} (SNIK\footnote{Hospital means \enquote{Krankenhaus} in German.}) is a modular OWL 2 DL ontology.
The \enquote{Meta Model} defines three basic disjunctive classes and their possible relations: Roles (who), Function (does what) and Entity Types (and which information is therefore needed).
A set of modular subontologies define subclasses of those three classes and their relations as described by a certain knowledge source about information management in hospitals:
%from different sources:% three textbooks, an interview and a standard.
The textbooks \citet{bb}, \citet{ob} and \citet{he} form the ontologies BB, OB and HE, respectively. 

\begin{table*}
\caption{Templates}
\label{tab:templates}
\begin{tabulary}{\textwidth}{lL}
\toprule
\textbf{Template}	&\textbf{Description}\\
\midrule
Definition		&Ask for the class that fits the given textbook definition.\\
Example			&What is defined as \enquote{Examination of in and out patients in radiological department}?\\
Distractors		&Labels of other classes (that have a path of length 2 or less to the correct class.)\\
\midrule
Subject			&Ask for the class that is related via a given relation to a given object.\\
Example			&Who is \emph{involved in} a \emph{healthcare network}?\\
Distractors		&Labels of other classes (of the same type) that \emph{are not} related via the same relation to the same object.\\
\midrule
Subject negative	&Ask for the class that is \emph{not} related via a given relation to a given object.\\
Example			&Who is \emph{involved in} a \emph{healthcare network}?\\
Distractors		&Labels of other classes (of the same type) that are related via the that relation to that object.\\
\midrule
Object			&Ask for the class that is related via a given relation from a given subject.\\ 
Example			&What is the \emph{CEO} \emph{responsible for}?\\
Distractors		&Labels of other classes that \emph{are not} related via the same relation from that subject\\
\midrule
Object negative		&Ask for the class that is \emph{not} related via a given relation from a given subject.\\ 
Example			&What is the \emph{CEO} \emph{responsible for}?\\
Distractors		&Labels of other classes that \emph{are} related via the same relation from that subject\\
\midrule
%Function		&Ask for the Function that is connected over a given relation (uses, updates, increases, decreases) to a given entity type.\\
%Example			&Which function \emph{uses} \emph{patient record}?\\
%Distractors		&\\
%\midrule
%Entity Type		&Which \\
%Example			&Which information does \emph{long term archiving} use?\\
%Distractors		&\\
%\midrule
Relationship		&Ask about the relationship between two classes.\\
Example			&How are \emph{Chief Information Officer} and \emph{Annual IT budget} related?\\
Distractors		&Other relations (with applicable domain and range) that do not connect the same subject and object\\
\midrule
Definition		&\\
Example			&\\
Distractors		&\\
\midrule
Definition		&\\
Example			&\\
Distractors		&\\
\bottomrule
\end{tabulary}
\end{table*}

\section{Discussion}
Because SNIK only contains the knowledge that the source textbooks describe, it does not contain the complete domain of Hospital Information Management.
As such, negative questions are problematic, as the given relationship could hold in the real world but not be described in the textbook source of the class. 
The same problem concerns the distractors of positive questions.
So this problem cannot be avoided.
However, this is one of the reasons that we don't ask count questions like \enquote{How many functions is the CIO responsible for?}, which don't help much for learning anyways. % improve writing

Relationships could also hold implicitly through the subclass hierarchy.
For example, \aurl{bb}{ChiefInformationOfficer} is a subclass of \aurl{bb}{InformationManagementStaff}, which is responsible for \aurl{bb}{OperationalInformationManagement}.
It is unclear, whether the CIO is also responsible for operational information management.

\section{Unsorted}
\citet{domaene} describes the structure of the inital meta model.

\section{Acknowledgments}
The SNIK project is supported by the DFG (German Research Foundation) under the grant numbers 1605/7-1 and 1387/8-1.
%\nocite{*} 
\bibliographystyle{ios1}
\bibliography{paper}
\end{document}
